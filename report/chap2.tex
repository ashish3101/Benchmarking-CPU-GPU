\chapter{Problems faced while Installation}

\section{How to install NVidia driver in Fedora 18}
-------------------------------------------------------------------------------------------------

Fedora 18 comes with open source \textbf{NOUVEAU driver} for NVIDIA graphics card. To install the NVIDIA propriteray driver follow the below steps.

\begin{enumerate}
\item Blacklist the nouveau driver: Add below line to /etc/modprobe.d/blacklist.conf file blacklist nouveau.
\item Rebuild the initramfs image file using dracut:\\
* Backup the initramfs file
\begin{verbatim}
$ sudo mv /boot/initramfs-(uname -r).img /boot/initramfs-(uname -r).img.bak
\end{verbatim}
* Rebuild the initramfs file
\begin{verbatim}
$ sudo dracut -v /boot/initramfs-(uname -r).img (uname -r)
\end{verbatim}
\item Reboot the system to runlevel 3 (without graphics)
\item Check that nouveau driver is not loaded
\begin{verbatim}
$ lsmod | grep nouveau
\end{verbatim}
\item Run the NVidia driver package
\begin{verbatim}
$ sudo ./NVIDIA-Linux-x86\_64-195.36.15-pkg2.run
\end{verbatim}
Above command will create xorg.conf file in /etc/X11 directory which is responsible to use NVidia driver in X.
\item Restart the system and NVidia driver will be used now.
\end{enumerate}

\section{How to install NVidia driver in Fedora 18}
-------------------------------------------------------------------------------------------------

\begin{enumerate}
\item Get the latest NVIDIA driver from here. Reboot with run-level 3 and install the driver. This should be pretty much straight-forward. Reboot after install.
\item Get the latest CUDA Toolkit.
\item Install the toolkit. This is a little bit tricky as the installer will claim that you have the wrong GCC version. In order to overcome this problem, run the installer with
\begin{verbatim}
>sh cuda_5.0.35_linux_64_fedora16-1.run -override compiler
\end{verbatim}
Skip the driver installation and select [y] for the toolkit and sample files.
\item Although in step (3) the GCC version check has been skipped, it is still hard-coded in one of the CUDA header files. Thus, go to line 80 of
\begin{verbatim}
/usr/local/cuda-5.0/include/host_config.h \end{verbatim} (or wherever the installer has put this header file) and change
\begin{verbatim}
#if __GNUC__ > 4 || (__GNUC__ == 4 && __GNUC_MINOR__ > 6)
    to
#if __GNUC__ > 4 || (__GNUC__ == 4 && __GNUC_MINOR__ > 7)
\end{verbatim}
Now NVCC won’t complain about this.
\item When you try to compile your project NVCC still claims about this
\begin{verbatim}
... atomicity.h(48): error: identifier "__atomic_fetch_add" is undefined
... atomicity.h(52): error: identifier "__atomic_fetch_add" is undefined
\end{verbatim}
We found a nice workaround for this problem here. All you need to do is
\begin{verbatim}
> echo "#undef _GLIBCXX_ATOMIC_BUILTINS" 
  > /usr/local/include/undef_atomics_int128.h
> echo "#undef _GLIBCXX_USE_INT128" 
  >> /usr/local/include/undef_atomics_int128.h
\end{verbatim}
and include this file when compiling with NVCC, i.e.
\begin{verbatim}
> nvcc.bin --pre-include /usr/local/include/undef_atomics_int128.h
  [other commands]
\end{verbatim}
Done. That worked for me, and I could compile the examples (after modifying the Makefiles with the “--pre-include” trick from (5)” without any troubles. Results from the NVIDIA samples look reasonable, what makes me think that CUDA 5.0 should work without (large) troubles on F18.\\
\begin{verbatim}
After following the steps above you may need to do the following
to build the examples.

A) The Makefiles don't pay attention to LD_LIBRARY_PATH or 
the ldconfig.
Assuming you installed to /usr/local, they use the path 
/usr/local/cuda/lib64 
but disregard the /usr/lib64/nvidia/ directory in which the libcuda*so
is located. To fix ad a symlink in the /usr/local/cuda/lib64 directory:
$ cd /usr/local/cuda/lib64
$ ln -s /usr/lib64/nvidia/libcuda.so libcuda.so

B) Run make from the /usr/local/cuda/samples directory as follows:
$ cd /usr/local/cuda/samples
$ make -k EXTRA_NVCCFLAGS=
"--pre-include /usr/local/include/undef_atomics_int128.h"
where /usr/local/include/undef_atomics_int128.h is the header file 
constructed in the previous post.
The "-k" tells make to continue even if a particular example fails to build,
e.g. for examples requiring something you don't have installed such as mpi. 
\end{verbatim}
\end{enumerate}
